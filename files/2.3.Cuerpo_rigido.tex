\documentclass[../Main.tex]{subfiles}

\begin{document}

\chapter{Cuerpo rigido}

\intro{
    A
    }

\npage{
% No hay imagenes en esta pagina
}{

    \section{Cinematica de un cuerpo rigido}
    Una condicion especial que tiene un cuerpo rigido es que al agarrar una
    particula $i$ y una particula $j$ se cumple que su distancia relativa es
    constante para todo momento.
    \begin{equation}
        |r_{ij} | = | r_i - r_j | = cte
        \label{eq:condrigidez}
    \end{equation}
    a esto se lo conoce como \textit{Condicion de Rigidez}.

    Ademas, un cuerpo rigido puede realizar dos tipos de movimieno diferentes.

    \begin{itemize}
        \item \textbf{Traslacion Pura:} Solo se traslada de posicion;
        \item \textbf{Rotacion Pura:} Existen un campo de velocidades y gira.
    \end{itemize}
    Y, obviamente, puede realizar una combinación de las dos, osea una \textbf{Roto transalacion}.

    Si tenemos en cuanta la ecuacion \ref{eq:condrigidez}, podemos decir que 
    \begin{equation}
        \vec{r_{ij}} = \vec{r_i} - \vec{r_j} \rightarrow \vec{v_{ij}} = \vec{v_i} - \vec{v_j}
        \label{eq:aux1}
    \end{equation}
    pero si lo vemos desde la velocidad
    \begin{equation*}
        \vec{v_{ij}} = \frac{\partial }{\partial t} \left( \vec{r_{ij}} \right) = \frac{\partial }{\partial t} \left( r_{ij} \cdot \hat{r}_{ij} \right)
    \end{equation*}
    si tenemos en cuenta la regla de la cadena de una derivada
    \begin{equation*}
        \vec{v_{ij}} = \frac{\partial }{\partial t} \left( r_{ij} \right) \cdot \hat{r}_{ij} + r_{ij} \cdot \frac{\partial }{\partial t} \left( \hat{r}_{ij} \right)
    \end{equation*}
    sabemos que la distancia entre los puntos no puede variar durante el tiempo
    ya que romperia con la condicion de rigidez, asi que solo nos queda el segundo
    termino
    \begin{equation*}
        \vec{v_{ij}} = r_{ij} \cdot \frac{\partial }{\partial t} \left( \hat{r}_{ij} \right) = r_{ij} \cdot \vec{\Omega} \times \hat{r}_{ij}
    \end{equation*}
    \begin{equation*}
        \Downarrow
    \end{equation*}
    \begin{equation*}
        \vec{v_{ij}} = \vec{\Omega} \times \vec{r}_{ij}
    \end{equation*}
    entonces si unimos con lo que habiamos conseguido en \ref{eq:aux1}
    \begin{equation*}
        \vec{v_{ij}} = \vec{v_i} - \vec{v_j} = \vec{\Omega} \times \vec{r}_{ij}
    \end{equation*}
    \begin{equation*}
        \Downarrow
    \end{equation*}

        }

\npage{
% No hay imagenes en esta pagina
}
{
    \begin{equation}
        \vec{v_i} = \vec{v_j} + \vec{\Omega} \times \left( \vec{r}_{i} - \vec{r}_{j} \right) \ .
    \end{equation}

    Partiendo de esta relacion podemos ver una condicion de las velocidades
    \begin{equation*}
        \vec{v_i} = \vec{v_j} + \vec{\Omega} \times \left( \vec{r}_{i} - \vec{r}_{j} \right)
    \end{equation*}
    \begin{equation*}
        \Downarrow
    \end{equation*}
    \begin{equation*}
        \vec{v_i} - \vec{v_j} = \vec{\Omega} \times \left( \vec{r}_{i} - \vec{r}_{j} \right)
    \end{equation*}
    \begin{equation*}
        \Downarrow
    \end{equation*}
    \begin{equation*}
        \left( \vec{v_i} - \vec{v_j} \right) \cdot \left( \vec{r}_{i} - \vec{r}_{j} \right) = \vec{\Omega} \times \left( \vec{r}_{i} - \vec{r}_{j} \right) \cdot \left( \vec{r}_{i} - \vec{r}_{j} \right)
    \end{equation*}
    la igualdad de la derecha es cero porque cuando uno hace el producto cruz
    consigue un ortogal de los vectores, si ademas hacemos el producto interno de 
    los factores eso va a dar como resultado cero ya que son ortogonales
    \begin{equation*}
        \left( \vec{v_i} - \vec{v_j} \right) \cdot \left( \vec{r}_{i} - \vec{r}_{j} \right) = 0
    \end{equation*}
    \begin{equation*}
        \Downarrow
    \end{equation*}
    \begin{equation*}
        \vec{v_i} \cdot \left( \vec{r}_{i} - \vec{r}_{j} \right) - \vec{v_j} \cdot \left( \vec{r}_{i} - \vec{r}_{j} \right) = 0
    \end{equation*}
    \begin{equation*}
        \Downarrow
    \end{equation*}
    \begin{equation*}
        \vec{v_i} \cdot \left( \vec{r}_{i} - \vec{r}_{j} \right) = \vec{v_j} \cdot \left( \vec{r}_{i} - \vec{r}_{j} \right)
    \end{equation*}
    \begin{equation*}
        \Downarrow
    \end{equation*}
    \begin{equation*}
        \vec{v_i} \cdot \hat{r}_{ij} = \vec{v_j} \cdot \hat{r}_{ij}
    \end{equation*}

    Ahora me gustaria mostrar si el omega del rigido es valido para todo el cuerpo
    rigido. Para eso voy a describir la velocidad desde tres puntos distintos
    \begin{equation*}
        \vec{v_i} = \vec{v_p} + \vec{\Omega} \times \left( \vec{r}_{i} - \vec{r}_{p} \right)
    \end{equation*}
    \begin{equation*}
        \vec{v_i} = \vec{v_q} + \vec{\Omega}' \times \left( \vec{r}_{i} - \vec{r}_{q} \right)
    \end{equation*}
    \begin{equation*}
        \vec{v_q} = \vec{v_p} + \vec{\Omega} \times \left( \vec{r}_{q} - \vec{r}_{p} \right)
    \end{equation*}

    donde $\vec{\Omega}'$ es otro omega del rigido que en principio no se si es
    igual al omega del rigido medido desde el punto $p$. El proximo paso es remplazar
    la ultima ecuacion en la segunda
    \begin{equation*}
        \vec{v_i} = \vec{v_p} + \vec{\Omega} \times \left( \vec{r}_{q} - \vec{r}_{p} \right) + \vec{\Omega}' \times \left( \vec{r}_{i} - \vec{r}_{q} \right)
    \end{equation*}
    ahora igualamos con la primera ecuacion
    \begin{equation*}
        \vec{v_p} + \vec{\Omega} \times \left( \vec{r}_{i} - \vec{r}_{p} \right) = \vec{v_p} + \vec{\Omega} \times \left( \vec{r}_{q} - \vec{r}_{p} \right) + \vec{\Omega}' \times \left( \vec{r}_{i} - \vec{r}_{q} \right)
    \end{equation*}

}

\npage{
    % No hay imagenes en esta pagina
}
{

    \begin{equation*}
        \vec{\Omega} \times \left( \vec{r}_{i} - \vec{r}_{p} \right) = \vec{\Omega} \times \left( \vec{r}_{q} - \vec{r}_{p} \right) + \vec{\Omega}' \times \left( \vec{r}_{i} - \vec{r}_{q} \right)
    \end{equation*}
    \begin{equation*}
        \Downarrow
    \end{equation*}
    \begin{equation*}
        \vec{\Omega} \times \left( \vec{r}_{i} - \vec{r}_{p} \right) - \vec{\Omega} \times \left( \vec{r}_{q} - \vec{r}_{p} \right) = \vec{\Omega}' \times \left( \vec{r}_{i} - \vec{r}_{q} \right)
    \end{equation*}
    \begin{equation*}
        \Downarrow
    \end{equation*}
    \begin{equation*}
        \vec{\Omega} \times \left( \left( \vec{r}_{i} - \vec{r}_{p} \right) - \left( \vec{r}_{q} - \vec{r}_{p} \right) \right) = \vec{\Omega}' \times \left( \vec{r}_{i} - \vec{r}_{q} \right)
    \end{equation*}
    \begin{equation*}
        \Downarrow
    \end{equation*}
    \begin{equation*}
        \vec{\Omega} \times \left( \vec{r}_{i} - \vec{r}_{q} \right) = \vec{\Omega}' \times \left( \vec{r}_{i} - \vec{r}_{q} \right)
    \end{equation*}
    \begin{equation*}
        \Downarrow
    \end{equation*}
    \begin{equation*}
        \vec{\Omega} \times \left( \vec{r}_{i} - \vec{r}_{q} \right) - \vec{\Omega}' \times \left( \vec{r}_{i} - \vec{r}_{q} \right) = 0
    \end{equation*}
    \begin{equation*}
        \Downarrow
    \end{equation*}
    \begin{equation*}
        \left( \vec{\Omega} - \vec{\Omega}' \right) \times \left( \vec{r}_{i} - \vec{r}_{q} \right) = 0
    \end{equation*}
    para que esto se cumpla, quiere decir que
    \begin{equation*}
        \vec{\Omega} = \vec{\Omega}'
    \end{equation*}

    \section{Eje instanteo de rotacion}
    Son los punto $\vec{r}_{j}$ que tienen $\vec{v}_j = 0$ en un dado instante.
    Si partimos de
    \begin{equation*}
        \vec{v}_i = \vec{v}_O + \vec{\Omega} \times \left( \vec{r}_i - \vec{r}_O \right)
    \end{equation*}
    y pedimos que $\vec{v}_0 = 0$ ya que el punto $O$ pertenece al eje de rotacion,
    tenemos que
    \begin{equation*}
        \vec{v}_i = \vec{\Omega} \times \left( \vec{r}_i - \vec{r}_O \right) \ .
    \end{equation*}

    Pero si ahora elegimos un punto $O'$ que tambien pertence al eje de rotacion,
    tenemos lo siguiente
    \begin{equation*}
        \vec{v}_{O'} = 0 = \vec{\Omega} \times \left( \vec{r}_{O'} - \vec{r}_O \right) \ .
    \end{equation*}
    esto quiere decir que el omega del rigido es paralelo al eje instantáneo de rotacion.

    \ejem{Varilla en un plano}{Dada una varilla que rueda sin deslizar, decir si
    existe un EIR.}

}

\npage{
    % No hay imagenes en esta pagina
}
{
    Escribamos la condicion de rigidez
    \begin{equation*}
    \vec{v}_{i} = \vec{v}_{CM} + \vec{\Omega} \times \left( \vec{r}_{i} - \vec{r}_{CM} \right) \ .
    \end{equation*}

    Ahora, en que gire haciendo rodadura quiere decir que la velicdad relativa
    entre la velocidad del piso y de la varilla es cero, por lo que ese punto tiene
    como velocidad cero, ya esto nos dice que existe un EIR.

    Ahora que sabemos que existe, me gustaria demostrar que existe una relacion
    entre la velocidad centro de masa y el omega del rigido

    \begin{equation*}
        \vec{v}_{i} = 0 = v_{CM} \cdot \hat{x} + \left( - \Omega \cdot \hat{z} \right) \times \left( 0 - r \cdot \hat{y} \right)
    \end{equation*}
    \begin{equation*}
        \Downarrow
    \end{equation*}
    \begin{equation*}
        0 = v_{CM} \cdot \hat{x} + \Omega \cdot r \cdot \hat{z} \times \hat{y}
    \end{equation*}
    \begin{equation*}
        \Downarrow
    \end{equation*}
    \begin{equation*}
        0 = v_{CM} \cdot \hat{x} + \Omega \cdot r \cdot \left( - \hat{x} \right)
    \end{equation*}
    \begin{equation*}
        \Downarrow
    \end{equation*}
    \begin{equation*}
        v_{CM} \cdot \hat{x} = \Omega \cdot r \cdot \hat{x}
    \end{equation*}

    Ahora que tenemos esta relacion, puedo demostrar una cosa interesante que
    aparece en las velocidades para estos casos 

    \begin{equation*}
        \vec{v}_{i} = \vec{v}_{O} + \vec{\Omega} \times \left( \vec{r}_{i} - \vec{r}_{O} \right)
    \end{equation*}
    si $O$ pertenece al EIR y tenemos nuestro origen en $0$
    \begin{equation*}
        \vec{v}_{i} = \vec{\Omega} \times \left( \vec{r}_{i} - 0 \right)
    \end{equation*}
    ademas, tenemos una relacion que une el omega
    \begin{equation*}
        \vec{v}_{i} = \frac{v_{cm}}{r} \times \vec{r}_{i}
    \end{equation*}
    con esta relacion podemos sacar un campo de velocidades y vemos que a medida
    que subimos, la velocidad es mayor.

    Agregar imagen de campos de velocidad

}

\npage{
    % No hay imagenes en esta pagina
}
{
    \section{Dinamica del cuerpo rigido}

    Para esta parte un sigue valiendo todo lo que se aprendio de dinamica de la
    primera parte e incluso siguen valiendo los teoremas de conservacion.

    Lo unico que cambia ahora es el hecho de obtener el centro de masa de un
    cuerpo rigido. Por definición, el centro de masa se calcula como
    \begin{equation*}
        \vec{r}_{CM} = \frac{\sum_i m_i \cdot \vec{r}_i}{\sum_i m_i}
    \end{equation*}
    pero ademas se puede ver como una integral que esta definida donde siempre
    hay masa, digo esto porque voy a escribir una integral que a priori pareciera 
    que no esta definida pero si lo esta solo que no escribo los limites
    \begin{equation*}
        \vec{r}_{CM} = \int \frac{ \partial m \cdot \vec{r}}{M}
    \end{equation*}
    M es la suma de todas las masas. Ademas, el $\partial m$ lo podemos ver como
    \begin{equation*}
        \partial m = \delta \cdot \partial V
    \end{equation*}
    donde $\delta$ es la densidad de masa y $\partial V$ es el diferencial volumen,
    osea en todas las direcciones donde el cuerpo rigido tenga volumen
    \begin{equation*}
        \vec{r}_{CM} = \int \frac{ \delta \left( \vec{r} \right) \cdot \partial V \cdot \vec{r}}{M}
    \end{equation*}

    \ejem{Cilindro Uniforme}{Dado un cilindro de radio $R$ y de altura $h$. Buscar
    el centro de masa.}

    Despues lo hago


}

\npage{
    % No hay imagenes en esta pagina
}
{
    \section{Impulso total de un cuerpo rigido}

    Ahora uno se podria preguntar si cambia calcular el impulso total desde otro
    punto en un cuerpo rigido.

    \begin{equation*}
        M \cdot \vec{v}_{CM} = \vec{P} = \sum_i m_i \cdot \vec{v}_i
    \end{equation*}
    \begin{equation*}
        \Downarrow
    \end{equation*}
    \begin{equation*}
        \vec{P} = \sum_i m_i \cdot \left( \vec{v}_O + \vec{\Omega} \times \left( \vec{r}_i - \vec{r}_O \right) \right)
    \end{equation*}
    \begin{equation*}
        \Downarrow
    \end{equation*}
    \begin{equation*}
        \vec{P} = M \cdot \vec{v}_O + \vec{\Omega} \times \left( M \cdot \vec{r}_{CM} - M \cdot \vec{r}_O \right)
    \end{equation*}

    Ahora nos podrimos preguntar que pasa con la rotacion. Si vemos el impulso
    angular medido desde el centro de masa tenemos

    \begin{equation*}
        \vec{L}_{CM} = \sum_i m_i \cdot \vec{r}_i \times \vec{v}_i = \int \partial m \cdot \vec{r} \times \vec{v} \left( \vec{r} \right)
    \end{equation*}
    donde $\vec{r}$ debe de estar medida desde el centro de masa. Ademas, por
    condicion de rigidez tenemos que
    \begin{equation*}
        \vec{v} \left( \ \vec{r} \ \right) = \vec{v}_{CM} + \vec{\Omega} \times \vec{r}
    \end{equation*}
    ahora podemos reemplazar en la ecuacion del impulso angular
    \begin{equation*}
        \vec{L}_{CM} = \int \partial m \cdot \vec{r} \times \left( \vec{v}_{CM} + \vec{\Omega} \times \vec{r} \right)
    \end{equation*}
    \begin{equation*}
        \Downarrow
    \end{equation*}
    \begin{equation*}
        \vec{L}_{CM} = \int \partial m \cdot \vec{r} \times \vec{v}_{CM} + \int \partial m \cdot \vec{r} \times \left( \vec{\Omega} \times \vec{r} \right)
    \end{equation*}
    a esto lo vamos a resolver de forma separada para que se entienda. Empezando por
    \begin{equation*}
        \int \partial m \cdot \vec{r} \times \vec{v}_{CM} = M \cdot \vec{r}_{CM} \times \vec{v}_{CM} = 0
    \end{equation*}
    esto es igual a cero porque estamos midiendo la distancia que hay del centro
    de masa respecto del centro de masa.
}
\npage{
    % No hay imagenes en esta pagina
}
{

    Despues tenemos
    \begin{equation*}
        \int \partial m \cdot \vec{r} \times \left( \vec{\Omega} \times \vec{r} \right)
    \end{equation*}
    el gran problema con este es el doble producto cruz. Para eso podemos pensarlo
    de la siguiente manera siguiendo la regla BACA-CABALLO que dice
    \begin{equation*}
        \vec{A} \times \left( \vec{B} \times \vec{C} \right ) = \vec{B} \cdot \left( \vec{A} \cdot \vec{C} \right) - \vec{C} \cdot \left( \vec{A} \cdot \vec{B} \right) 
    \end{equation*}
    entonces hagamos el doble producto cruz aparte y ademas tenemos en cuenta que
    el producto cruz de adentro, solo importa la parte que es ortgonal de $\vec{r}$
    \begin{equation*}
        \vec{r} \times \left( \vec{\Omega} \times \vec{r} \right) = \vec{\Omega} \cdot \left( \vec{r} \cdot \vec{r}_{\bot} \right) - \vec{r}_{\bot} \cdot \left( \vec{r} \cdot \vec{\Omega} \right)
    \end{equation*}
    \begin{equation*}
        \Downarrow
    \end{equation*}
    \begin{equation*}
        \vec{r} \times \left( \vec{\Omega} \times \vec{r} \right) = \vec{\Omega} \cdot r_{\bot}^2 - \vec{r}_{\bot} \cdot r_{\diagup \diagup} \cdot \Omega
    \end{equation*}
    entonces reemplzamos los que sabemos
    \begin{equation*}
        \vec{L}_{CM} = 0 + \int \partial m \left( \vec{\Omega} \cdot r_{\bot}^2 - \vec{r}_{\bot} \cdot r_{\diagup \diagup} \cdot \vec{\Omega} \right)
    \end{equation*}
    \begin{equation*}
        \Downarrow
    \end{equation*}
    \begin{equation*}
        \vec{L}_{CM} = \int \vec{\Omega} \cdot r_{\bot}^2 \cdot \partial m  - \int \vec{r}_{\bot} \cdot r_{\diagup \diagup} \cdot \Omega \cdot \partial m
    \end{equation*}
    \begin{equation*}
        \Downarrow
    \end{equation*}
    \begin{equation*}
        \vec{L}_{CM} = \vec{\Omega} \int \cdot r_{\bot}^2 \cdot \partial m  - \int \vec{r}_{\bot} \cdot r_{\diagup \diagup} \cdot \Omega \cdot \partial m
    \end{equation*}
    en particular a 
    \begin{equation*}
        I_{CM} = \int r_{\bot}^2 \cdot \partial m
    \end{equation*}
    se lo conoce como \textit{Momento de inercia} medido desde el centro de masa
    \begin{equation*}
        \vec{L}_{CM} = \vec{\Omega} \cdot I_{CM} - \vec{L}_{\bot, CM} \ ,
    \end{equation*}

    Ahora nos quedaria preguntar cuando el $\vec{L}_{\bot, CM} = 0$.

    Si $\vec{\Omega}$ es paralelo al eje de simetria se cumple que

}

\npage{
    % No hay imagenes en esta pagina
}
{

    \begin{equation*}
        \partial \vec{L}_{\bot, 1} + \partial \vec{L}_{\bot, 2} = 0
    \end{equation*}
    donde
    \begin{equation*}
        \partial \vec{L}_{\bot, 1} = \partial m \cdot \vec{r}_{\bot , 1} \cdot r_{\diagup \diagup , 1}
    \end{equation*}
    \begin{equation*}
        \partial \vec{L}_{\bot, 2} = \partial m \cdot \vec{r}_{\bot , 2} \cdot r_{\diagup \diagup , 2}
    \end{equation*}

    para este caso se cumple que
    \begin{equation*}
        r_{\diagup \diagup , 1} = r_{\diagup \diagup , 2}
    \end{equation*}
    y que
    \begin{equation*}
        | \vec{r}_{\bot , 1} | = | \vec{r}_{\bot , 2} |
    \end{equation*}
    pero son de sentidos contrarios
    \begin{equation*}
        \vec{r}_{\bot , 1} =  - \vec{r}_{\bot , 2}
    \end{equation*}
    por eso se anulan.

    El otro caso es cuando existe un plano de simetria, en este caso pasa al revez,
    la parte ortogonal es igual pero la parte paralela es contraria.

    \section{Teorema de Steiner}

    Ahora me gustaria demostrar la relacion que existe entre el $I_z$ con un $I_{z'}$.
    Si partimos de la expresión que encontramos en la anterior seccion
    \begin{equation*}
        I_{z'} = \int r_{\bot}^2 \cdot \partial m
    \end{equation*}
    \begin{equation*}
        \Downarrow
    \end{equation*}
    \begin{equation*}
        I_{z'} = \int \left[ x'^2 + y'^2 \right] \partial m
    \end{equation*}
    Ademas podemos decir que solo se mueve en el eje $y$ por lo que su valor en 
    $x$ se mantedra pero su valor en $y$ tendra una diferencia
    \begin{equation*}
        \Downarrow
    \end{equation*}
    \begin{equation*}
        I_{z'} = \int \left[ x^2 + (y - \alpha )^2 \right] \partial m
    \end{equation*}
    \begin{equation*}
        \Downarrow
    \end{equation*}
    \begin{equation*}
        I_{z'} = \int \left[ x^2 + y^2 - 2 \cdot y \cdot \alpha + \alpha ^2 \right] \partial m
    \end{equation*}
    \begin{equation*}
        \Downarrow
    \end{equation*}
    \begin{equation*}
        I_{z'} = \int \left( x^2 + y^2 \right) \partial m + \int \left( - 2 \cdot y \cdot \alpha \right) \partial m + \int \left( \alpha ^2 \right) \partial m
    \end{equation*}

}
\npage{
    % No hay imagenes en esta pagina
}
{
    \begin{equation*}
        I_{z'} = I_{z} + -2 \cdot \alpha \int \left( y \right) \partial m + M \cdot \alpha^2
    \end{equation*}
    donde
    \begin{equation*}
        -2 \cdot \alpha \int \left( y \right) \partial m = M \cdot y_{cm} = 0
    \end{equation*}
    ya que mis sistema de referncia esta parado en el centro de masa. Entonces
    nos que da que
    \begin{equation*}
        I_{z'} = I_{z} + M \cdot \alpha^2
    \end{equation*}
    donde $\alpha$ es la distancia entre los ejes.
}
\end{document}
