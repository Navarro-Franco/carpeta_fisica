\documentclass[../Main.tex]{subfiles}

\begin{document}
\chapter{Cinemática del punto}

\intro{
    Muchas veces nos interesa dar una descripcion del movimiento de un objeto
sin entrar en las caracteristicas de este. Un ejemplo, a muy grandes rasgos, es
el GPS de nuestro teléfono. Un GPS sencillo no puede distinguir si esta
describiendo el movimiento de una persona corriendo, de un auto o de tren, e
incluso no le interesa si las ruedas son de cierto material o patinan respecto
del piso. Por eso, podemos reducir el cuerpo que estamos tratando de describir
a un solo punto y ver como este se traslada en funcion del tiempo.
}

%\section{Vectores}
%A mi me resulta muy necesario primero empezar con el tema de vectores, mas que 
%nada su forma de escribir. Estamos de acuerdo que lo presente en la figura
%es propiamente un vector. Se puede ver que tiene un sentido, una magnitud y

\section{Movimiento Rectilíneo}
Para el estudio del movimiento de un cuerpo puntual uno hace la siguiente
preposicion y es que este adopta una sola posicion en cada instante. Con esta
preposicion estamos diciendo que el movimiento de nuestro cuerpo va a tener las
caracteristicas de una funcion pero lo que no se dijo es de que va a depender 
la funcion posicion.

Al menos yo, cuando leo la palabra cinematica me imagino una serie de fotos
sacadas consecutivamente en un intervalo de tiempo corto que si uno las pudiera
unir, veria lo que muchos conocemos como un video. Y en si la funcion posicion
eso. Son una serie de punto en un determinado tiempo que al unirnos nos dicen
como un objeto mueve durante un determinado intervalo de tiempo.

Ahora, la funcion de tiempo $\va{r}(t)$ puede tener cualquier aspecto de funcion
que uno se imagine (poner imagenes ) y no solo eso sino que en los movimientos
que uno quiera, por ejemplo uno podria tener una
$\va{r}(t) = x(t) \cdot \hat{x} + y(t) \cdot \hat{y} + z(t) \cdot \hat{z} $
donde la posicion varia en las tres dimensiones a la vez.

\thmp{Velocidad y movimiento en un MRU}{
\begin{equation}
    v(t) = cte
\end{equation}
\begin{equation}
    x(t) = x_0 + v \cdot (t - t_0)
\end{equation}
}{
    Se parte de la base que la velocidad es constante para todo tiempo y que
    la velocidad es la derivada de la distancia respecto del tiempo.
\begin{equation}
    v = \frac{\partial x}{\partial t}
\end{equation}
    Se puede pasar el diferencia al otro lado e integrar en un intervalo de
    tiempo determinado.
\begin{equation}
    v \cdot \partial t = \partial x \Longrightarrow \int _{t_0}^t v \cdot \partial t = \int _{x_0}^{x(t)} \partial x \Longrightarrow v \cdot (t - t_0) = x(t) - x_0
\end{equation}
}

\thmp{Velocidad y movimiento en un MRUV}{
\begin{equation}
    a = cte
\end{equation}
\begin{equation}
    v(t) = v_0 + a \cdot (t - t_0)
\end{equation}
\begin{equation}
    x(t) = x_0 + v_0 \cdot (t - t_0) + \frac{a}{2} \cdot (t - t_0)^2
\end{equation}
}{
    Se parte de la base que la velocidad es constante para todo tiempo y que
    la velocidad es la derivada de la distancia respecto del tiempo.
\begin{equation}
    v = \frac{\partial x}{\partial t}
\end{equation}
    Se puede pasar el diferencia al otro lado e integrar en un intervalo de
    tiempo determinado.
\begin{equation}
    v \cdot \partial t = \partial x \Longrightarrow \int _{t_0}^t v \cdot \partial t = \int _{x_0}^{x(t)} \partial x \Longrightarrow v(t) \cdot (t - t_0) = x(t) - x_0
\end{equation}
}

\end{document}
